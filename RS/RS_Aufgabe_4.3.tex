\documentclass[a4paper,11pt]{article}
\title{Aufgabe 4.3}
\begin{document}
\maketitle
\section*{(a)}
$8,626 \cdot 10^5 = 0.08626 \cdot 10^7 \\
\\
9,94420 + 0,08626 = 10,03046\\
\\
10,03046 \cdot 10^7 = 1,003046 \cdot 10^8 \approx 1,003 \cdot 10^8$

\section*{(b)}
$8,626 \cdot 10^5 \approx 0.0862 \cdot 10^7 \\
\\
9,9442 + 0,0862 = 10,0304\\
\\
10,0304 \cdot 10^7 \approx 1,0030 \cdot 10^8$

\section*{(c)}
In diesem Beispiel gibt es zwar keinen Unterschied im Endergebnis, aber in anderen F\"allen kann es durch Rundungen der Zwischenergebnisse zu geringen Ungenauigkeiten kommen. Allerdings sorgen gerundete (und damit kleinere) Zwischenergebnisse daf\"ur, dass der Rechenaufwand sinkt. Au\ss erdem werden Digitalrechner die Zwischenergebnisse ebenfalls runden, falls das Endergebnis sp\"ater auch gerundet werden soll, da alle Zahlen in gleich gro\ss en Speichern liegen.\\ 
Welches Verfahren bevorzugt werden sollte, liegt aber endg\"ultig daran, wie genau das Ergebnis werden soll.
\end{document}