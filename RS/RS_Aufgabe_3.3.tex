\documentclass[paper, a4]{article}
\title{Aufgabe 3.3}


\begin{document}
\maketitle
Divisionsrestverfahren:\\
Falls eine Division einen Rest hat, muss das Ergebnis um $1$ erh\"oht werden (z.B. 2. Zeile "$-15$" statt "$-14$"), da der "Rest" beim R\"uckschluss addiert werden muss. D.h. in der 2. Zeile w\"urde ($-14 \cdot 2 = -28; -28 + 1($Rest$) = -27$) nicht das gleiche sein, wie der Startwert "$-29$".\\
\\
$-58 : 2 = -29$ "Rest" +0 \\
$-29 : 2 = -15$ "Rest" +1 \\
$-15 : 2 =\hspace{5pt} -8$ "Rest" +1 \\
$~-8 : 2 =\hspace{5pt} -4$ "Rest" +0 \\
$~-4 : 2 =\hspace{5pt} -2$ "Rest" +0 \\
$~-2 : 2 =\hspace{5pt} -1$ "Rest" +0 \\
$~-1 : 2 =\hspace{5pt} -1$ "Rest" +1 ~$\uparrow$ Leserichtung\\
\\
$\rightarrow 1000110$\\
\\
Da wir eine 8-Bit zahl ben\"otigen, aber nur 7 Rest-Zahlen haben, m\"ussen wir Einsen hinzuf\"ugen, bis wir 8 Stellen haben.\\
\\
$\rightarrow 11000110$\\
\\
Diese Zahl entspricht der Zahl, die in Aufgabe 3.2 berechnet wurde.
\end{document}