\documentclass[a4paper,11pt]{article}

\usepackage[T1]{fontenc}
\usepackage{titling}

\setlength{\droptitle}{-11em}

\title{L\"osung Aufgabenblatt 1}

\begin{document}
\maketitle
\textbf{Aufgabe 1.3}\\
\section*{(a)}
Es liegt vor: $y = (a \cdot x^6 + b \cdot x^5 + c \cdot x^4 + d \cdot x^3 + e \cdot x^2 + f \cdot x + g) $\\
	Bei einem Polynom sechsten Grades benötigt man 5 Multiplikationen zur Bildung der Potenzen $x^6, x^5, x^4, x^3, x^2$ und 6 weitere zur Multiplikation der Potenzen mit ihren Koeffizienten.\\
	Aus den sechs Ergebnissen der Multiplikationen ergeben sich zusammen mit $g$ nun zus\"atzlich 6 Additionen, um das Polynom zu l\"osen.
	\paragraph{}
	Wenn ein CPU f\"ur eine Multiplikation 6 ns und f\"ur eine Addition 1 ns braucht, ergibt sich f\"ur die Dauer der Berechnung des Polynoms:\\
	$6 ns  \cdot 11 + 1 ns \cdot 6 = 72 ns$ 
	\paragraph{}
	Bei der Umformung des Polynoms durch das Horner-Schema liegt folgendes vor:\\
	$y = ((((((a \cdot x + b) \cdot x + c) \cdot x + d) \cdot x + e) \cdot x + f) \cdot x +g)$\\
	In dieser Form benötigt man immer noch 6 Additionen, allerdings ergeben sich durch die Aufl\"osung der Potenzen nur noch 6 Multiplikationen.
	\paragraph{}
	Wenn der CPU das Polynom mit Hilfe des Horner-Schemas l\"ost, ergibt sich:\\
	$6ns \cdot 6 + 1ns \cdot 6 = 42ns$
\section*{(b)}
Es liegt vor: $y = (x^2+2x+1)^{11}$\\
Die geringste Anzahl von Multiplikationen erreicht man durch die Umformung zu: $y=((x+2) \cdot x+1)^{11}$\\
Rechenschritte: \\
$(x+2)=a \\ (a\cdot x)=b \\ (b+1)=c \\ c^{11}=d$\\
Es werden 2 Multiplikationen und 2 Additionen ben\"otigt.\\
F\"ur die CPU Zeit ergibt sich: $6ns \cdot 2 + 1ns \cdot 2= 14ns$

\end{document}