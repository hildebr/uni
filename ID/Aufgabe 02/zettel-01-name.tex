\documentclass[a4paper,10pt]{article}

% Hier die Nummer des Blatts und Autoren angeben.
\newcommand{\blatt}{2}
\newcommand{\autor}{Alexander Hildebrandt}

\usepackage{hci}

\begin{document}
% Seitenkopf mit Informationen
\kopf
\renewcommand{\figurename}{Figure}

\aufgabe{2}


Das Thought Paper ''The Model Human Processor: An Engineering Model of Human Performance'' von S.K. Card, T.P. Morian und A. Newell beschaeftigt sich mit einem psychologischem Modell, das sowohl die Aufnahme und Verarbeitung von Informationen als auch die Ausfuehrung von Aktionen an Hand von eben diesen Informationen beschreibt.
Der "Model Human Processor" wird beschrieben als ein Set von Prozessoren, dem Gedaechtnis und deren Verbindungen zusammen mit einem Set von Operationsprinzipien. Außerdem werden die drei Gedaechtnis-Parameter Speicherkapazitaet, Halbwertszeit und Code-Typ (ikonisch, akustisch, visuell oder semantisch) von Wahrnehmungsobjekten festgelegt. Dazu gibt es noch den Prozessor-Parameter Taktzeit.
Laut dem MHP besteht der Verstand aus einem perzeptuellem, einem kognitiven und einem motorischen Prozessor, die alle ca. 100ms Taktzeit haben. Diese Prozessoren koennen je nach Aufgabe seriell oder parallel operieren. Perzeptuelle Ereignisse, die in einem Takt auftreten, werden in ein Wahrnehmungsobjekt kombiniert, die daraufhin im Arbeitsgedaechtnis bereitstehen und durch den kognitiven Prozessor verarbeitet werden. In jedem Takt werden die Inhalte des Arbeitsgedaechtnisses erkannt und eine verbundene Aktion aus dem Langzeitgedaechtnisses wird eingeleitet. Dadurch wird das Arbeitsgedaechtniss modifiziert. Zuletzt fuehrt der motorische Prozessor Bewegungen aus, die durch vorangegangene Prozesse bestimmt sind. Diese Bewegungen sind nicht wie man denken koennte fluessig, sondern bestehen aus einer Serie von Mikrobewegungen.
Durch die drei MHP Versionen ''Slowman'', ''Fastman'' und ''Middleman'' werden Randfaelle und Durchschnitt in Taktzeit dargestellt und viele vorangegangene Experimente zu Reaktionszeiten bewahrheiten dadurch das MHP.

\end{document}
