\documentclass[12pt,a4paper]{../krautsourcing/homework}
\usepackage[utf8]{inputenc}
\usepackage[ngerman]{babel}
\usepackage[T1]{fontenc}
\usepackage{amsmath}
\usepackage{graphicx}
\usepackage{amsfonts}
\usepackage{amssymb}
\usepackage{lmodern}
\usepackage{amsmath}
\usepackage{amssymb}
\usepackage{paralist}
\usepackage{tabularx}
\usepackage{tikz}
\usetikzlibrary{automata,positioning}

\author{Ruben Felgenhauer,\\Alexander Hildebrandt,\\Leonhard Reichenbach}
\datef{18}{10}{2015}
\course{Formale Grundlagen der Informatik II}
\sheet{2}
\sectionprefix{Übungsaufgabe \thesheet.}
\subsectionprefix{\thesheet.}
\subsubsectioncounter{\alph{subsubsection}}
\group{06}
\subsubsectionprefix{(}
\subsubsectionsuffix{)}

\begin{document}

\makeheadline

\addtocounter{section}{2}

\section{}

\subsection{}
\begin{align*}
L(A_n) = &\left((ab)^0d(ba)^0 + \ldots + (ab)^{\frac{n}{2}}d(ba)^{\frac{n}{2}}\right)  \\
+ &\left((ab)^0ada(ba)^0 + \ldots + (ab)^{\frac{n}{2}-1}ada(ba)^{\frac{n}{2}-1}\right)\\
+ &\left( ab \right)^{\frac{n}{2}}
\end{align*}

\subsection{}
\begin{align*}
L(A_n) = & \{ab\}^0\cdot\{d\}\cdot\{ba\}^0 \cup \ldots \cup \{ab\}^{\frac{n}{2}}\cdot\{d\}\cdot\{ba\}^{\frac{n}{2}}  \\
 \cup &\{ab\}^0\cdot\{ada\}\cdot\{ba\}^0 \cup \ldots \cup \{ab\}^{\frac{n}{2}-1}\cdot\{ada\}\cdot\{ba\}^{\frac{n}{2}-1}\\
\cup & \{ab\} ^{\frac{n}{2}}
\end{align*}

\subsection{}
\begin{align*}
Z.zg.: L(A_n) = & \{ab\}^0\cdot\{d\}\cdot\{ba\}^0 \cup \ldots \cup \{ab\}^{\frac{n}{2}}\cdot\{d\}\cdot\{ba\}^{\frac{n}{2}}  \\
 \cup &\{ab\}^0\cdot\{ada\}\cdot\{ba\}^0 \cup \ldots \cup \{ab\}^{\frac{n}{2}-1}\cdot\{ada\}\cdot\{ba\}^{\frac{n}{2}-1}\\
\cup & \{ab\} ^{\frac{n}{2}}
\end{align*}
Diesen Block nennen wir im folgenden der Übersicht halber \(M(A_n)\)\\
\(\)

\begin{tabularx}{\linewidth}{@{}>{\bfseries}l@{\hspace{.5em}}X@{}}
    \(\glqq\Rightarrow\grqq:\) &
    Sei \(w \in L(A_n) \).

    Dann wird \(w\) von \(A_n\) akzeptiert. \(A_n\) hat zwei Endzustände \(Z_1\) und \(Z_{2n}\). Nun gibt es drei Möglichkeiten welche Form \(w\) haben kann.
    
    Form 1:
    
    Um den Endzustand \(Z_{2n}\) zu erreichen muss \(w\) aus \(\frac{n}{2}\) vielen Aneinanderreihungen von \(ab\) bestehen. Also \(w = \{ab\}^{\frac{n}{2}} \) und damit auch \(w \in M(A_n)\).
    
    Form 2:
    
    Um den Endzustand \(Z_1\) zu erreichen gibt es zwei Möglichkeiten, hier die erste. Der direkte Weg zu \(Z_1\) ist immer über \(w=d\) vorhanden, für \(n \geq 2 \) kommt nun die Möglichkeit hinzu im Automaten einen Bogen zu \glqq laufen\grqq . Das funktioniert wie folgt: zuerst liest man bis zu \(\frac{n}{2}\) viele \(ab\), dann ein \(d\) um nach \glqq unten\grqq {} zu kommen und anschließend liest man \(\frac{n}{2}\) viele \(ba\) um in \(Z_1\) zu landen. \(w\) kann also alle Formen zwischen \(\{ab\}^0 \cdot \{d\} \cdot\{ba\}^0 \) und \(\{ab\}^{\frac{n}{2}} \cdot \{d\} \cdot\{ba\}^{\frac{n}{2}} \) annehmen. Auch hier gilt \(w \in M(A_n)\).
    
    Form 3:
    
    Funktioniert analog zu Form 2: zuerst liest man bis zu \(\frac{n}{2}-1\) viele \(ab\), dann noch ein \(a\), dann \(d\) um nach \glqq unten\grqq {} zu kommen und anschließend liest man noch ein \(a\) und \(\frac{n}{2}-1\) viele \(ba\) um in \(Z_1\) zu landen. \(w\) kann also alle Formen zwischen \(\{ab\}^0 \cdot \{ada\} \cdot\{ba\}^0 \) und \(\{ab\}^{\frac{n}{2}-1} \cdot \{ada\} \cdot\{ba\}^{\frac{n}{2}-1} \) annehmen. Auch hier gilt wieder \(w \in M(A_n)\).

    Also ist \(w \in M(A_n)\) und \(L(A_n) \subseteq M(A_n)\).

     \\

    \(\glqq\Leftarrow\grqq:\) &
    Sei \(w \in M(A_n)\).

    Wir unterteilen \(w\) in drei Fälle.
    
    Fall 1:
    
    \(w = \{ab\}^{\frac{n}{2}}\)
    
    Daraus ergibt sich folgende Kantenrelation:
    
    \( 
    \delta (Z_0, \quad \{ab\}^{\frac{n}{2}} ) \mapsto 
    \delta (Z_2, \quad \{b\}\cdot \{ab\}^{\frac{n}{2}-1} ) \mapsto
    \delta (Z_4, \quad \{ab\}^{\frac{n}{2}-1} ) \mapsto
    \ldots \mapsto 
    \delta (Z_{2n}, \lambda) 
    \)

    D.h. bei Eingaben dieser Form landet der Automat immer in \(Z_{2n}\). Dies ist ein Endzustand.
    
    Fall 2:
    
    \(w \in \{ab\}^0\cdot\{d\}\cdot\{ba\}^0 \cup \ldots \cup \{ab\}^{\frac{n}{2}}\cdot\{d\}\cdot\{ba\}^{\frac{n}{2}} \)
    
    Dann hat \(w\) die Form \(w = u\cdot \{d\} \cdot v \mid u = \{ab\}^x \text{ und } v = \{ba\}^x \text{ wobei } x \in (0, \frac{n}{2}) \)
    
    Daraus ergibt sich folgende Kantenrelation:
    
    \(
    \delta (Z_0, \quad u \cdot \{d\} \cdot v) \mapsto
    \ldots \mapsto
    \delta (Z_{4x}, \quad \{d\} \cdot v) \mapsto
    \delta (Z_{4x+1}, \quad v) \mapsto
    \ldots \mapsto
    \delta (Z_1, \lambda)
    \)    
    
    D.h. bei Eingaben dieser Form landet der Automat immer in \(Z_1\). Dies ist ein Endzustand.
    
    Fall 3:
    
    \(w \in \{ab\}^0\cdot\{ada\}\cdot\{ba\}^0 \cup \ldots \cup \{ab\}^{\frac{n}{2}-1}\cdot\{ada\}\cdot\{ba\}^{\frac{n}{2}-1} \)
    
    Dann hat \(w\) die Form \(w = u\cdot \{ada\} \cdot v \mid u = \{ab\}^x \text{ und } v = \{ba\}^x \text{ wobei } x \in (0, \frac{n}{2}-1) \)
    
    Daraus ergibt sich folgende Kantenrelation:
    
    \(
    \delta (Z_0, \quad u \cdot \{ada\} \cdot v) \mapsto
    \ldots \mapsto
    \delta (Z_{4x}, \quad \{ada\} \cdot v) \mapsto
    \delta (Z_{4x+2}, \quad \{da\} \cdot v) \mapsto
    \delta (Z_{4x+3}, \quad \{a\} \cdot v) \mapsto
    \delta (Z_{4x+1}, \quad v) \mapsto
    \ldots \mapsto
    \delta (Z_1, \lambda)
    \)    
    
    D.h. auch bei Eingaben dieser Form landet der Automat immer in \(Z_1\). Dies ist ein Endzustand.

	Also ist \(w \in L(A_n)\) und \(M(A_n) \subseteq L(A_n)\).	  
\end{tabularx}

\hfill \(\square\)


\subsection{}
Die Sprache \(L(A_n)\) ist regulär, da sie als regulärer Ausdruck geschrieben werden kann.\\ Siehe 1.3.1

%\subsection{} keine Lust mehr auf Zusatz..

\section{}

\subsection{}

\begin{compactenum}
\item Die Menge der Start- und Endzustände wird vertauscht bzw.
\(Q'_0 := F \text{ und } F' := \lbrace q_0 \rbrace\)
\item Alle Kanten werden umgekehrt bzw. \( \delta' := \lbrace (p,w,q) \mid (q,w,p) \in \delta \rbrace \)
%bzw. \( \forall (q, w, p) \in \delta : \exists  (p, w, q) \in \delta' \) \\
\item Aus dem nun entstandenen NFA wird mittels Potenzautomatenkonstruktion ein DFA erstellt.
\item Der entstandene DFA wird ggf. vollständig gemacht.
\end{compactenum}



\subsection{}
Z.zg.: \(L(A) = \Sigma^* \cdot \{reed\} \cdot \Sigma^* \mid \Sigma = \lbrace r,e,d \rbrace \) \\
\(\)

\begin{tabularx}{\linewidth}{@{}>{\bfseries}l@{\hspace{.5em}}X@{}}
    \(\glqq\Rightarrow\grqq:\) &
    Sei \(w \in L(A) \).

    Dann wird \(w\) von \(A\) akzeptiert. Dazu muss \(w\) in \(q_4\) enden, da dies der einzige Endzustand ist. Am Anfang ist der Automat in \(q_0\). Um zu \(q_4\) zu gelangen, muss man durch die restlichen drei Zustände gehen. Die einzige Zeichenkette, die Richtung \(q_4\) führt ist \(reed\). Falls dieses Wort mit anderen Buchstaben unterbrochen wird, geht man zurück in Richtung \(q_0\). Sobald man in \(q_4\) angekommen ist, kann man alle Zeichen in $\Sigma$ lesen und bleibt im Endzustand. D.h. \(w\) hat die Form  \( u \cdot reed \cdot v \mid u,v \in \Sigma^*\).

    Also ist \(w \in \Sigma^* \cdot \{reed\} \cdot \Sigma^* \) und \(L(A') \subseteq\Sigma^* \cdot \{reed\} \cdot \Sigma^*\).

     \\

    \(\glqq\Leftarrow\grqq:\) &
    Sei \(w \in \Sigma^* \cdot \{reed\} \cdot \Sigma^*\).

    Dann gilt \(w = x_1 \cdot x_2 \cdot x_3 \mid x_1,x_3 \in \Sigma^* \quad \text{und} \quad x2 = reed \).
    
    Daraus ergibt sich folgende Kantenrelation:
    
    \( 
    \delta (q_0, \quad x_1 \cdot x_2 \cdot x_3) \mapsto 
    \delta (\{q_0,q_1,q_2,q_3,q_4 \}, \quad x_2 \cdot x_3) \mapsto
    \delta (q_4, \quad x_3) \mapsto 
    \delta (q_4, \lambda) 
    \)

    D.h. bei Eingaben dieser Form landet der Automat immer in \(q_4\). \(q_4\) ist ein Endzustand.

	Also ist \(w \in L(A)\) und \(\Sigma^* \cdot \{reed\} \cdot \Sigma^* \subseteq L(A')\).	  
\end{tabularx}

\hfill \(\square\)

\subsection{}
Der reguläre Ausdruck zu \(\Sigma^* \cdot \{reed\} \cdot \Sigma^*\) lautet \(\Sigma^* \cdot reed \cdot \Sigma^*\).

\subsection{}
\begin{compactenum}
\item Der Ursprüngliche Automat: \\
\begin{tikzpicture}[node distance=3cm,on grid,auto]
	\node[state,initial,initial text={}] (q0)               {\(q_0\)};
	\node[state]                         (q1) [right of=q0] {\(q_1\)};
	\node[state]                         (q2) [right of=q1] {\(q_2\)};
	\node[state]                         (q3) [right of=q2] {\(q_3\)};
	\node[state,accepting]               (q4) [right of=q3] {\(q_4\)};
	\path[->]
		(q0) edge [loop above] node [above] {\(d,e\)}   ()
		(q0) edge [bend left]  node [above] {\(r\)}     (q1)
		(q1) edge [loop above] node [above] {\(r\)}     ()
		(q1) edge [bend left]  node [below] {\(d\)}     (q0)		
		(q1) edge [bend left]  node [above] {\(e\)}     (q2)
		(q2) edge [bend left]  node [below] {\(r\)}     (q1)
		(q2) edge              node [above] {\(e\)}     (q3)
		(q3) edge              node [above] {\(d\)}     (q4)
		(q4) edge [loop above] node [above] {\(r,e,d\)} ()
	; %-path-%
	\draw[->](q3) to[out=120,  in=60,  edge node={node [above] {\(r\)}}] (q1);
    \draw[->](q2) to[out=-120, in=-60, edge node={node [below] {\(d\)}}] (q0);
	\draw[->](q3) to[out=-90,  in=-90, edge node={node [below] {\(e\)}}] (q0);
\end{tikzpicture}



\item Kanten umkehren: \\
\begin{tikzpicture}[node distance=3cm,on grid,auto]
	\node[state,accepting]                                   (q0)               {\(q_0\)};
	\node[state]                                             (q1) [right of=q0] {\(q_1\)};
	\node[state]                                             (q2) [right of=q1] {\(q_2\)};
	\node[state]                                             (q3) [right of=q2] {\(q_3\)};
	\node[state,initial,initial text={},initial where=right] (q4) [right of=q3] {\(q_4\)};
	\path[->]
		(q0) edge [loop above] node [above] {\(d,e\)}   ()		
		(q1) edge [bend right] node [above] {\(r\)}     (q0)
		(q1) edge [loop above] node [above] {\(r\)}     ()
		(q0) edge [bend right] node [below] {\(d\)}     (q1)
		(q2) edge [bend right] node [above] {\(e\)}     (q1)		
		(q1) edge [bend right] node [below] {\(r\)}     (q2)						
		(q3) edge              node [above] {\(e\)}     (q2)
		(q4) edge              node [above] {\(d\)}     (q3)
		(q4) edge [loop above] node [above] {\(r,e,d\)} ()
	; %-path-%
	\draw[->](q1) to[out=60,  in=120,  edge node={node [above] {\(r\)}}] (q3);
	\draw[->](q0) to[out=-60, in=-120, edge node={node [below] {\(d\)}}] (q2);
	\draw[->](q0) to[out=-90, in=-90,  edge node={node [below] {\(e\)}}] (q3);
\end{tikzpicture}

\newpage

\item Potenzautomaten konstruieren:\\
A': \\
\begin{tikzpicture}[node distance=3cm,on grid,auto]
	\node[state,accepting]                                   (q0q1q2q3q4)                       {\(\lbrace q_0,q_1,q_2,q_3,q_4\rbrace\)};
	\node[state]                                             (q1q4)       [right of=q0q1q2q3q4] {\(\lbrace q_1,q_4\rbrace\)};
	\node[state]                                             (q2q4)       [right of=q1q4]       {\(\lbrace q_2,q_4\rbrace\)};
	\node[state]                                             (q3q4)       [right of=q2q4]       {\(\lbrace q_3,q_4\rbrace\)};
	\node[state,initial,initial text={},initial where=right] (q4)         [right of=q3q4]       {\(\lbrace q_4\rbrace\)};
	\path[->]
		(q0q1q2q3q4) edge [loop above] node [above] {\(r,e,d\)} ()
		(q1q4)       edge              node [above] {\(r\)}     (q0q1q2q3q4)
		(q2q4)       edge              node [above] {\(e\)}     (q1q4)
		(q3q4)       edge              node [above] {\(e\)}     (q2q4)
		(q3q4)       edge [bend right] node [below] {\(r,d\)}   (q4)
		(q4)         edge [loop above] node [above] {\(r,e\)}   ()	
		(q4)         edge [bend right] node [above] {\(d\)}     (q3q4)
	; %-path-%	
	\draw[->] (q2q4) to[out=60,  in=120,  edge node={node [below] {\(r,d\)}}] (q4);
	\draw[->] (q1q4) to[out=-60, in=-120, edge node={node [above] {\(e,d\)}}] (q4);
\end{tikzpicture}
\item Vollständig machen: Der Automat ist bereits vollständig!
\end{compactenum}


\subsection{}

Z.zg.: \(L(A') = \lbrace w^{rev} \mid w \in L(A) \rbrace = \lbrace w^{rev} \mid w \in \Sigma^* \cdot \{reed\} \cdot \Sigma^* \rbrace = \Sigma^* \cdot \{deer\} \cdot \Sigma^* \) \\
\(\)

\begin{tabularx}{\linewidth}{@{}>{\bfseries}l@{\hspace{.5em}}X@{}}
    \(\glqq\Rightarrow\grqq:\) &
    Sei \(w \in L(A') \).

    Dann wird \(w\) von \(A\) akzeptiert. Dazu muss \(w\) in \(q_0\) enden, da dies der einzige Endzustand ist. Am Anfang ist der Automat in \(q_4\). Um zu \(q_0\) zu gelangen, muss man durch die restlichen drei Zustände gehen. Die einzige Zeichenkette, die Richtung \(q_0\) führt ist \(deer\). Falls dieses Wort mit anderen Buchstaben unterbrochen wird, geht man zurück in Richtung \(q_4\). Sobald man in \(q_0\) angekommen ist, kann man alle Zeichen in $\Sigma$ lesen und bleibt im Endzustand. D.h. \(w\) hat die Form  \( u \cdot deer \cdot v \mid u,v \in \Sigma^*\)

    Also ist \(w \in \Sigma^* \cdot \{deer\} \cdot \Sigma^* \) und \(L(A') \subseteq \lbrace w^{rev} \mid w \in L(A) \rbrace\).

     \\

    \(\glqq\Leftarrow\grqq:\) &
    Sei \(w \in \Sigma^* \cdot \{deer\} \cdot \Sigma^*\).

    Dann gilt \(w = x_1 \cdot x_2 \cdot x_3 \mid x_1,x_3 \in \Sigma^* \quad \textrm{und} \quad x_2 = deer \).
    
    Daraus ergibt sich folgende Kantenrelation:
    
    \( 
    \delta (q_4, \quad x_1 \cdot x_2 \cdot x_3) \mapsto 
    \delta (\{q_0,q_1,q_2,q_3,q_4 \}, \quad x_2 \cdot x_3) \mapsto
    \delta (q_0, \quad x_3) \mapsto 
    \delta (q_0, \lambda) 
    \)

    D.h. bei Eingaben dieser Form landet der Automat immer in  \(q_0\). \(q_0\) ist ein Endzustand.

	Also ist \(w \in L(A')\) und \(\lbrace w^{rev} \mid w \in L(A) \rbrace \subseteq L(A')\).	  
\end{tabularx}

\hfill \(\square\)

\end{document}
