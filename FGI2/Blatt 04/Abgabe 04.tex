\documentclass[12pt,a4paper]{../krautsourcing/homework}
\usepackage[utf8]{inputenc}
\usepackage[ngerman]{babel}
\usepackage[T1]{fontenc}
\usepackage{amsmath}
\usepackage{graphicx}
\usepackage{amsfonts}
\usepackage{amssymb}
\usepackage{lmodern}
\usepackage{paralist}
\usepackage{tabularx}
\usepackage{tikz}
\usetikzlibrary{calc}
\usepackage{amstext}
\usepackage{array}
\usepackage{mathtools}

\author{Ruben Felgenhauer,\\Alexander Hildebrandt,\\Leonhard Reichenbach}
\datef{08}{11}{2015}
%\date{\today}
\course{Formale Grundlagen der Informatik II}
\sheet{4}
\sectionprefix{Übungsaufgabe \thesheet.}
\subsectionprefix{\thesheet.}
\subsubsectioncounter{\alph{subsubsection}}
\group{06}
\subsubsectionprefix{(}
\subsubsectionsuffix{)}

\newcommand{\TScell}{TS_\textit{cell}}
\newcommand{\Mcell}{M_\textit{cell}}
\newcommand{\eLocked}{\textit{Locked}}
\newcommand{\eBattery}{\textit{Battery}}
\newcommand{\eOn}{\textit{On}}
\newcommand{\eActive}{\textit{Active}}
\newcommand{\eError}{\textit{Error}}
\newcommand{\nmodels}{\not\models}

%% <ltl-kram>
%% <here be dragons>
\newcommand{\DeclareTLOperator}[2]{%
	\DeclareMathOperator{#1}{%
%% </here be dragons>
%% <tweaking area>
			\newcommand{\width}{8pt} % Breite und Höhe
			\newcommand{\tlraise}{-0.45pt} % Grundlinienversatz
			\newcommand{\preindent}{-2mu} % Einzug vor dem OP
			\newcommand{\postindent}{-4mu} % Einzug nach dem OP
			\tikzset{ % Styleset für die \draw-Funktion
				tlstyle/.style={
					line width=0.5pt
				}
			}
%% </tweaking area>		
%% <here be dragons>
			\mspace{\preindent}\raisebox{\tlraise}{%
				\tikz{%
					\coordinate (tl) at (0.00,   \width);
					\coordinate (tr) at (\width, \width);
					\coordinate (br) at (\width, 0.00);
					\coordinate (bl) at (0.00,   0.00);
					\coordinate (t)  at ($(tl)!0.5!(tr)$); 
					\coordinate (r)  at ($(tr)!0.5!(br)$);
					\coordinate (b)  at ($(bl)!0.5!(br)$);
					\coordinate (l)  at ($(tl)!0.5!(bl)$);
					\draw[tlstyle] #2 --cycle;
				}
			}\mspace{\postindent}
	}
}
\DeclareTLOperator{\tlG}{%
	(tl) to[out=0,   in=180] (tr)
		 to[out=270, in=90]  (br)
		 to[out=180, in=0]   (bl)
		 to[out=90,  in=270] (tl)
}
\DeclareTLOperator{\tlF}{%
	(t)  to[out=315, in=135] (r)
		 to[out=225, in=45]  (b)
		 to[out=135, in=315] (l)
		 to[out=45,  in=225] (t)
}
\DeclareTLOperator{\tlX}{%
	(t)	 to[out=0,   in=90]  (r)
		 to[out=270, in=0]   (b)
		 to[out=180, in=270] (l)
		 to[out=90,  in=180] (t)
}
%% </here be dragons>
%% </ltl-kram>

\begin{document}

\makeheadline

\addtocounter{section}{2}

\section{}

\subsection{}

\begin{align*}
	L(\TScell) &= (ou((lu)^* + (c t^* h))^* \cdot (f + c t^* e o^* r s))^*
	\\
	L^\omega(\TScell) &= (ou((lu)^* + (c t^* h))^* \cdot (f + c t^* e o^* r s))^\omega
\end{align*}

\subsection{}

\begin{align*}
	SS(\Mcell) &= 0(12((12)+(3^+2))^* \cdot (0 + 3^+4^+50))^\omega
\end{align*}

\subsection{}

\begin{align*}
	ES(c_0) =&\ \{\eLocked,\eBattery\}\\
	ES(c_1) =&\ \{\eLocked,\eBattery,\eOn\}\\
	ES(c_2) =&\ \{\eBattery,\eOn\}\\
	ES(c_3) =&\ \{\eBattery,\eOn,\eActive\}\\
	ES(c_4) =&\ \{\eLocked,\eBattery,\eError\}\\
	ES(c_5) =&\ \{\eLocked\}\\
	ES(SS(\Mcell)) =&\ ES(c_0)(ES(c_1)ES(c_2)((ES(c_1)ES(c_2))+\\&\ (ES(c_3)^+ES(c_2)))^* \cdot (ES(c_0) + ES(c_3)^+ES(c_4)^+ES(c_5)ES(c_0)))^\omega
\end{align*}

\subsection{}

\begin{align*}
	Sat(\eError) &= \{c_4\}
	\\ Sat(\lnot \eBattery) &= \{c_5\}
	\\ Sat(\eOn) &= \{c_1,c_2,c_3\}
\end{align*}
Immer, wenn ein Fehler auftritt, gilt, dass wenn man im nächsten Schritt die Batterie entfernt, dass das Telefon dann zukünftig wieder an sein wird.\\
\begin{tabularx}{\linewidth}{@{}>{\bfseries}l@{\hspace{.5em}}X@{}}
Behauptung: &
Die Formel \(f\) gilt in \(c_0\).\\
Beweis: & Einen Error erreicht man von \(c_0\) nur, indem man über \(c_1\) \(c_2\) und 
\(c_3\) nach \(c_4\) übergeht. In \(c_4\) kann der nächste Schritt auf \(c_5\) gehen. In diesem Fall gilt nun \(\lnot \eBattery\). Von \(c_5\) kann man nur nach \(c_0\) übergehen, von da aus nur nach \(c_1\). In \(c_1\) gilt wieder \(On\).
\end{tabularx}

\subsection{}

\begin{tabularx}{\linewidth}{@{}>{\bfseries}l@{\hspace{.5em}}X@{}}
Behauptung: &
Die Formel \(g\) gilt in \(c_0\).\\
Beweis: & Einen Error erreicht man von \(c_0\) nur, indem man über \(c_1\) \(c_2\) und \(c_3\) nach \(c_4\) übergeht. In \(c_4\) kann der nächste Schritt auf \(c_5\) gehen, von \(c_5\) nur nach \(c_0\), von da aus nur nach \(c_1\), von  und von da aus nur nach \(c_2\). Hier existiert nun ein Weg nach \(c_3\), in dem \(\eActive\) gilt.
\end{tabularx}
Ein Pfad \(\pi\) lautet beispielsweise:
\begin{align*}
	\pi = c_0 c_1 c_2 c_3 c_4 c_5 c_0 c_1 c_2 c_3
\end{align*}

\section{}

\subsection{}

\renewcommand{\arraystretch}{1.2} 
\centerline{\begin{tabular}{|c|>{\(}c<{\)}|c|c|}
\hline
\# & f & \(\Mcell \models f\) & \(\Mcell,\pi \models f\) \\
\hline
1 & \tlG(\eActive)
& Falsch & Falsch \\
2 & \tlG\tlF(\eActive)
& Falsch & Wahr \\
3 & \tlG(\tlX \eActive \Rightarrow \eOn)
& Wahr & Wahr \\
4 & \tlG\tlF(\eActive \Rightarrow \tlX\tlX \lnot \eOn)
& Wahr & Wahr \\
5 & \tlG\tlF(\lnot \eBattery \lor \eActive \lor \lnot \eOn \lor \eError)
& Falsch & Wahr \\
6 & \tlX\tlX\tlX \eActive
& Wahr & Wahr \\
\hline
\end{tabular}}
\vspace{5mm}
\begin{enumerate}
\item
\(\Mcell \nmodels \tlG(\eActive)\), denn bereits im Startzustand \(c_0\) \(\lnot \eActive\) gilt.
\\
\(\Mcell, \pi \nmodels \tlG(\eActive)\) aus dem gleichen Grund.
\item
\(\Mcell \nmodels \tlG\tlF(\eActive)\), da mit den Pfad \(\pi' \coloneqq (c_0 c_1 c_2 )^\omega\) gilt \(\Mcell, \pi' \models \tlG(\lnot \eActive)\).
\\
\(\Mcell, \pi \models \tlG\tlF(\eActive)\), da in \(c_3\) stets \(\eActive\) gilt.
\item
\(\Mcell \models \tlG(\tlX \eActive \Rightarrow \eOn)\), da \(\eActive\) nur in \(c_3\) gilt, und in den vorangehenden Zuständen \(c_2\) und \(c_3\) \(\eOn\) gilt. Außerdem wird der geklammerte Audruck immer wahr, falls \((\tlX\lnot\eActive)\) gilt.
\\
\(\Mcell, \pi \models \tlG(\tlX \eActive \Rightarrow \eOn)\) aus dem gleichen Grund.
\item \(\Mcell \models \tlG\tlF(\eActive \Rightarrow \tlX\tlX \lnot \eOn)\), da \((\eActive \Rightarrow \tlX\tlX \lnot \eOn)\) immer wahr wird, wenn \(\lnot \eActive\) gilt. Da dies schon in \(c_0\) der Fall ist, gilt die Aussage auf allen Pfaden.
\\
\(\Mcell, \pi \models \tlG\tlF(\eActive \Rightarrow \tlX\tlX \lnot \eOn)\) aus dem gleichen Grund.
\item
\(\Mcell \nmodels \tlG\tlF(\lnot \eBattery \lor \eActive \lor \lnot \eOn \lor \eError)\), da für \(\pi' \coloneqq (c_0 (c_1 c_2)^\omega)\) nach dem Startzustand nie wieder \((\lnot \eBattery \lor \eActive \lor \lnot \eOn \lor \eError)\) gilt.
\\
\(\Mcell,\pi \models \tlG\tlF(\lnot \eBattery \lor \eActive \lor \lnot \eOn \lor \eError)\), da hier \(c_0\) immer wieder durchlaufen wird und hier \((\lnot \eBattery \lor \eActive \lor \lnot \eOn \lor \eError)\) gilt.
\item
\(\Mcell \nmodels \tlX\tlX\tlX \eActive\), da mit \(\pi' \coloneqq (c_0c_1c_2)^\omega\) niemals \(\eActive\) gilt.
\\
\(\Mcell, \pi \models \tlX\tlX\tlX \eActive\), da von \(c_0\) aus startend drei Schritte weiter \(c_3\) folgt, und hier \(\eActive\) gilt.
\end{enumerate}

\subsection{}

\subsubsection{}

\(\tlG (\lnot \eBattery \Rightarrow \lnot \eOn) \rightarrow True\)

\subsubsection{}

\(\tlG \tlF (\eOn) \rightarrow True \)

\subsubsection{}

%$\tlG (\eError \Rightarrow (\eActive \Rightarrow XError)) \rightarrow True $

\(\tlG (\tlX \eError \Rightarrow \eActive) \rightarrow True\)

\subsubsection{}

\(\tlG (\eOn \lor \eError \lor \tlX \eOn) \rightarrow \text{ Gegenbeispiel: } c_5\)

\end{document}