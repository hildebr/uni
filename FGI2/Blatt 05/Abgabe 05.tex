\documentclass[12pt,a4paper]{../krautsourcing/homework}
\usepackage[utf8]{inputenc}
\usepackage[ngerman]{babel}
\usepackage[T1]{fontenc}
\usepackage{amsmath}
\usepackage{graphicx}
\usepackage{amsfonts}
\usepackage{amssymb}
\usepackage{lmodern}
\usepackage{amsmath}
\usepackage{amssymb}
\usepackage{paralist}
\usepackage{tabularx}
\usepackage{setspace} 
\usepackage{tikz}
\usepackage{mathtools}
\usepackage{array}
\usetikzlibrary{automata,positioning,calc,backgrounds}
\usepackage{ifthen}

\author{Ruben Felgenhauer,\\Alexander Hildebrandt,\\Leonhard Reichenbach}
\datef{15}{11}{2015}
%\date{\today}
\course{Formale Grundlagen der Informatik II}
\sheet{5}
\sectionprefix{Übungsaufgabe \thesheet.}
\subsectionprefix{\thesheet.}
\subsubsectioncounter{\alph{subsubsection}}
\group{06}
\subsubsectionprefix{(}
\subsubsectionsuffix{)}

%% <ltl-kram>
%% <here be dragons>
\newcommand{\tlcheckargs}[1]{
	\ifthenelse{\equal{#1}{letter}\OR\equal{#1}{symbol}}{}{%
    	\errmessage{Operator style must be either "symbol" or "letter"}
    }
}
\newcommand{\thetlstyle}{symbol}
\newcommand{\tlstyle}[1]{%
	\tlcheckargs{#1}
	\renewcommand{\thetlstyle}{#1}
}
\newcommand{\DeclareTLOperator}[3]{%
	\newcommand{#1}[1][\thetlstyle]{%
		\tlcheckargs{##1}
		\mathop{%
			\ifthenelse{\equal{#3}{}\OR\equal{##1}{letter}}{%
				\textbf{#2}
			}{%
%% </here be dragons>
%% <tweaking area>			
				\newcommand{\width}{8pt}
				\newcommand{\baseline}{-0.3pt}
				\newcommand{\preindent}{-2mu}
				\newcommand{\postindent}{-4mu}	
				\tikzset{ 
					tldrawstyle/.style={
						line width=0.5pt
					}
				}
%% </tweaking area>		
%% <here be dragons>				
				\mspace{\preindent}\raisebox{\baseline}{%
					\tikz{%
						\coordinate (bl) at (0,0);
						\coordinate (tr) at (\width,\width);
						\coordinate (tl) at (bl|-tr);
						\coordinate (br) at (tr|-bl);
						\coordinate (t)  at ($(tl)!0.5!(tr)$); 
						\coordinate (r)  at ($(tr)!0.5!(br)$);
						\coordinate (b)  at ($(bl)!0.5!(br)$);
						\coordinate (l)  at ($(tl)!0.5!(bl)$);
						\draw[tldrawstyle] #3 --cycle;
					}
				}\mspace{\postindent}
			}
		}
	}
}
\DeclareTLOperator{\tlG}{G}{%
	(tl) to[out=0,   in=180] (tr)
		 to[out=270, in=90]  (br)
		 to[out=180, in=0]   (bl)
		 to[out=90,  in=270] (tl)
}
\DeclareTLOperator{\tlF}{F}{%
	(t)  to[out=315, in=135] (r)
		 to[out=225, in=45]  (b)
		 to[out=135, in=315] (l)
		 to[out=45,  in=225] (t)
}
\DeclareTLOperator{\tlX}{X}{%
	(t)	 to[out=0,   in=90]  (r)
		 to[out=270, in=0]   (b)
		 to[out=180, in=270] (l)
		 to[out=90,  in=180] (t)
}
\DeclareTLOperator{\tlU}{U}{}
\DeclareTLOperator{\tlE}{E}{}
\DeclareTLOperator{\tlA}{A}{}
%% </here be dragons>
%% </ltl-kram>

\newcommand{\br}{\\ \phantom{-} \\}
\newcommand{\as}[1]{\textcolor{blue}{\footnotesize{#1}}}
\newcommand{\nodesize}{3cm}
\tikzset{
	statex/.style={
		state,
		minimum width=20mm,
		align=center,
		scale=1.2
	}
}
\newcommand{\drawtscell}[7]{\scalebox{0.65}{\begin{tikzpicture}[auto,baseline]
	\node[statex,initial left,initial text={},accepting]  (c0) {#2};
	\node[above right=-2mm of c0]                              {\(c_0\)};
	\node[statex,below right=40mm and 80mm of c0]         (c4) {#6};
	\node[above left=-1mm of c4]                               {\(c_4\)};
	\node[statex,at=(c0|-c4)]                             (c2) {#4};
	\node[above left=-1mm of c2]                               {\(c_2\)};
	\node[statex,at=(c4|-c0)]                             (c5) {#7};
	\node[above left=-1mm of c5]                               {\(c_5\)};
	\node[statex,at=($(c2)!0.5!(c4)$)]                    (c3) {#5};
	\node[above right=-2mm of c3]                              {\(c_3\)};
	\node[statex,at=($(c0)!0.5!(c3)$)]                    (c1) {#3};
	\node[above right=-2mm of c1]                              {\(c_1\)};
	\node[above left=15mm of c0,align=left,anchor=west]        {(Teil-)Formel: #1};
	\path[->]
		(c0) edge                            node [below left]              {\(o\)}
		                                     node [above right=-2mm]        {\as{turn-on}}         (c1)
		(c1) edge [bend right=20]            node [right]                   {\(u\)}
		                                     node [above=3mm]               {\as{unlock}}          (c2)
		(c2) edge                            node [below left]              {\(f\)}
		                                     node [above left]              {\as{turn-off}}        (c0)
		(c2) edge [bend right=20]            node [left]                    {\(l\)}
		                                     node [above right=2mm and 3mm] {\as{lock}}            (c1)
		(c2) edge [bend right=20]            node [above]                   {\(c\)}
		                                     node [below]                   {\as{call}}            (c3)
		(c3) edge [bend right=20]            node [below]                   {\(h\)}
		                                     node [above right=-2mm]        {\as{hang-up}}         (c2)
		(c3) edge [loop above,distance=15mm] node [below]                   {\(t\)}
		                                     node [above]                   {\as{talk}}            (c3)
		(c3) edge                            node [above]                   {\(e\)}
		                                     node [below]                   {\as{error-occurence}} (c4)
		(c4) edge [loop right,distance=15mm] node [left]                    {\(o\)}
		                                     node [above=3mm]               {\as{turn-on}}         (c4)
		(c4) edge                            node [right]                   {\(r\)}
		                                     node [left=-1mm]               {\as{remove-battery}}  (c5)
		(c5) edge                            node [above]                   {\(s\)}
		                                     node [below]                   {\as{set-battery}}     (c0)
	; %-path-%
\end{tikzpicture}}}

\newcommand{\TScell}{TS_\textit{cell}}
\newcommand{\Mcell}{M_\textit{cell}}
\newcommand{\eLocked}{\textit{Locked}}
\newcommand{\eBattery}{\textit{Battery}}
\newcommand{\eOn}{\textit{On}}
\newcommand{\eActive}{\textit{Active}}
\newcommand{\eError}{\textit{Error}}
\newcommand{\nmodels}{\not\models}
\newcommand{\true}{\text{True}}

\begin{document}

\makeheadline

\addtocounter{section}{2}

\section{}
\subsection{}
\vspace{-5mm}
\tlstyle{letter}
\begin{align*}
f &= \lnot (\tlE \true \tlU \lnot (\lnot \eError \lor \tlE ( \eError \tlU \lnot \eBattery)) \lor \lnot \tlE \tlG \lnot \eActive)
\\
g &= \lnot \tlE \true \tlU ( \lnot \tlE\relax [\true \tlU \eActive])
\end{align*}
\vspace{-10mm}
\subsubsection{}
\hspace*{-12mm}
\begin{tabular}{c>{\hspace{-10mm}}c<{}}
\drawtscell
	{\(f_0 \coloneqq \eActive\)}
	{}
	{}
	{}
	{\(f_0\)}
	{}
	{}
&
\drawtscell
	{\(f_1 \coloneqq \tlE \tlG (\lnot \eActive)\)}
	{\(f_1\)}
	{\(f_1\)}
	{\(f_1\)}
	{\(f_0\)}
	{\(f_1\)}
	{\(f_1\)}
\br
\drawtscell
	{\(f_2 \coloneqq \lnot (\tlE \tlG (\lnot \eActive))\)}
	{\(f_1\)}
	{\(f_1\)}
	{\(f_1\)}
	{\(f_0\)}
	{\(f_1\)}
	{\(f_1\)}
&
\drawtscell
	{\(f_3 \coloneqq \lnot \eBattery\)}
	{\(f_1\)}
	{\(f_1\)}
	{\(f_1\)}
	{\(f_0\)}
	{\(f_1\)}
	{\(f_1\) \br \(f_3\)}
\end{tabular}
\newpage
\hspace*{-20mm}
\begin{tabular}{c>{\hspace{-10mm}}c<{}}
\drawtscell
	{\(f_4 \coloneqq \tlE ( \eError \tlU \lnot \eBattery)\)}
	{\(f_1\)}
	{\(f_1\)}
	{\(f_1\)}
	{\(f_0\)}
	{\(f_1\) \br \(f_4\)}
	{\(f_1\) \br \(f_3\) \br \(f_4\)}
&
\drawtscell
	{\(f_5 \coloneqq \lnot (\lnot \eError \lor \tlE ( \eError \tlU \lnot \eBattery))\)}
	{\(f_1\)}
	{\(f_1\)}
	{\(f_1\)}
	{\(f_0\)}
	{\(f_1\) \br \(f_4\)}
	{\(f_1\) \br \(f_3\) \br \(f_4\)}
\br
\drawtscell
	{\(f_6 \coloneqq \tlE \true \tlU \lnot (\lnot \eError \lor \tlE ( \eError \tlU \lnot \eBattery))\)}
	{\(f_1\)}
	{\(f_1\)}
	{\(f_1\)}
	{\(f_0\)}
	{\(f_1\) \br \(f_4\)}
	{\(f_1\) \br \(f_3\) \br \(f_4\)}
&
\drawtscell
	{\(f = \lnot (\tlE \true \tlU \lnot (\lnot \eError \lor \tlE ( \eError \tlU \lnot \eBattery)) \lor \lnot \tlE\tlG \lnot \eActive)\)}
	{\(f_1\) \br \(f\)}
	{\(f_1\) \br \(f\)}
	{\(f_1\) \br \(f\)}
	{\(f_0\) \br \(f\)}
	{\(f_1\) \br \(f_4\) \br \(f\)}
	{\(f_1\) \br \(f_3\) \br \(f_4\) \br \(f\)}
\end{tabular}

\subsubsection{}

\hspace*{-14mm}
\begin{tabular}{c>{\hspace{-10mm}}c<{}}
\drawtscell
	{\(g_0 \coloneqq \eActive\)}
	{}
	{}
	{}
	{\(g_0\)}
	{}
	{}
&
\drawtscell
	{\(g_1 \coloneqq \tlE \true \tlU \eActive\)}
	{\(g_1\)}
	{\(g_1\)}
	{\(g_1\)}
	{\(g_0\) \br \(g_1\)}
	{\(g_1\)}
	{\(g_1\)}
\end{tabular}

\newpage

\hspace*{-20mm}
\begin{tabular}{c>{\hspace{-10mm}}c<{}}
\drawtscell
	{\(g_2 \coloneqq \lnot  \tlE \true \tlU \eActive\)}
	{\(g_2\)}
	{\(g_2\)}
	{\(g_2\)}
	{\(g_0\) \br \(g_2\)}
	{\(g_2\)}
	{\(g_2\)}
&
\drawtscell
	{\(g_3 \coloneqq \tlE \true \tlU (\lnot \tlE\relax[\true \tlU \eActive])\)}
	{\(g_2\)}
	{\(g_2\)}
	{\(g_2\)}
	{\(g_0\) \br \(g_2\)}
	{\(g_2\)}
	{\(g_2\)}
\br
\drawtscell
	{\(g = \lnot \tlE \true \tlU (\lnot \tlE\relax[\true \tlU \eActive])\)}
	{\(g_2\) \br \(g\)}
	{\(g_2\) \br \(g\)}
	{\(g_2\) \br \(g\)}
	{\(g_0\) \br \(g_2\) \br \(g\)}
	{\(g_2\) \br \(g\)}
	{\(g_2\) \br \(g\)}
\end{tabular}

\subsection{}

\begin{tabularx}{\linewidth}{@{}>{\(}l<{\):}@{\hspace{.5em}}X@{}}
f &
Auf allen Pfaden gilt immer, dass wenn \eError\ gilt, \eError\ dann so lange gilt, bis \eBattery\ nicht mehr gilt.
Außerdem gilt auf nicht allen Pfaden irgendwann \eActive.
\\
g &
Für alle Pfade gilt immer, dass es einen Pfad gibt, auf dem irgendwann einmal \eActive\ gilt.
\end{tabularx}

\subsection{}

\begin{align*}
	\text{Sat}(\phi) = \text{Sat}(f) = \text{Sat}(g) = S
\end{align*}

\subsection{}
Wie in 5.3.1 bewiesen, gilt \(\forall c \in S(M) : M, c \models f\), somit gilt auch \(M, c_0 \models f\).
 
\newpage

\section{}

\subsection{}

\begin{tikzpicture}[auto, baseline, node distance=25mm,statex/.style={state,draw,minimum width=15mm}]
\node                                                  (a1)  {\(A_1\):};
\node[statex,right=15mm of a1,initial,initial text={}] (r1)  {\(\{r_1\}\)};
\node[statex,right of=r1]                              (rg1) {\(\{rg_1\}\)}; 
\node[statex,right of=rg1]                             (gr1) {\(\{(gr_1\}\)}; 
\node[statex,right of=gr1]                             (g1)  {\(\{g_1\}\)}; 
\node[below=30mm of a1]                                (a2)  {\(A_2\):};
\node[statex,initial,initial text={}, at=(r1|-a2)]     (gr2) {\(\{gr_2\}\)};
\node[statex,right of=gr2]                             (g2)  {\(\{g_2\}\)};
\node[statex,right of=g2]                              (r2)  {\(\{r_2\}\)};
\node[statex,right of=r2]                              (rg2) {\(\{rg_2\}\)};
\path[->]
	(r1)  edge               (rg1)
	(rg1) edge               (gr1)
	(gr1) edge               (g1)
    (gr2) edge               (g2)
    (g2)  edge               (r2)
	(r2)  edge               (rg2)
	(g1)  edge[bend left=40] (r1)
	(rg2) edge[bend left=40] (gr2)
; %-path-%
\end{tikzpicture}

\subsection{}

\begin{tikzpicture}[auto, baseline, node distance=25mm,statex/.style={state,draw,minimum width=19mm}]
\node                                                            (ts)    {\(A_1\otimes A_2\):};
\node[statex,initial,initial text={},accepting,right=10mm of ts] (r1gr2) {\(\{r_1, gr_2\}\)};
\node[statex,right of=r1gr2]                                     (r1g2)  {\(\{r_1, g_2\}\)};
\node[statex,right of=r1g2]                                      (rg1r2) {\(\{rg_1, r_2\}\)};
\node[statex,right of=rg1r2]                                     (gr1r2) {\(\{gr_1, r_2\}\)};
\node[statex,right of=gr1r2]                                     (g1r2)  {\(\{g_1, r_2\}\)};
\node[statex,right of=g1r2]                                      (r1rg2) {\(\{r_1, rg_2\}\)};
\path[->]
	(r1gr2) edge                (r1g2)
	(r1g2)  edge                (rg1r2)
	(rg1r2) edge                (gr1r2)
	(gr1r2) edge                (g1r2)
	(g1r2)  edge                (r1rg2)
	(r1rg2) edge [bend left=30] (r1gr2)
; %-path-%
\end{tikzpicture}


\subsection{}

\begin{align*}
	\lnot \phi = \lnot (\tlG \lnot (gr_1 \land gr_2)) = \tlF (gr_1 \land gr_2)
\end{align*}
\begin{tikzpicture}[auto, baseline]
\node (ts) {\(M_{\lnot \phi}\):};

\node[state,initial,initial text={},right=10mm of ts] (nil) {\(m_0\)};
\node[state,accepting,right=25mm of nil]              (all) {\(m_1\)};
\path[->]
	(nil) edge[loop above] node[above] {\(\lnot \{ gr_1, gr_2\}\)}
	(nil) edge             node[above] {\(\{gr_1, gr_2\}\)}        (all)
	(all) edge[loop above] node[above] {\(w\)}                     (all)
; %-path-%
\end{tikzpicture}

\newpage

\subsection{}

\begin{align*}
\text{ES}(A_1\otimes A_2) = (\{r_1,gr_2\}\cdot\{r_1,g_2\}\cdot\{rg_1,r_2\}\cdot\{gr_1,r_2\}\cdot\{g_1,r_2\}\cdot\{r_1,rg_2\})^\omega
\end{align*}
\begin{tikzpicture}[auto, baseline, node distance=20mm]
\node                                                           (ts) {\(B\):};
\node[state,initial,initial text={},accepting,right=10mm of ts] (b0) {\(b_0\)};
\node[state,right of=b0]                                        (b1) {\(b_1\)};
\node[state,right of=b1]                                        (b2) {\(b_2\)};
\node[state,right of=b2]                                        (b3) {\(b_3\)};
\node[state,right of=b3]                                        (b4) {\(b_4\)};
\node[state,right of=b4]                                        (b5) {\(b_5\)};
\path[->]
	(b0) edge[bend right]    node[below] {\(\{r_1,gr_2\}\)} (b1)
	(b1) edge[bend right]    node[below] {\(\{r_1,g_2\}\)}  (b2)
	(b2) edge[bend right]    node[below] {\(\{rg_1,r_2\}\)} (b3)
	(b3) edge[bend right]    node[below] {\(\{gr_1,r_2\}\)} (b4)
	(b4) edge[bend right]    node[below] {\(\{g_1,r_2\}\)}  (b5)
	(b5) edge[bend right=25] node[above] {\(\{r_1,rg_2\}\)} (b0)
; %-path-%
\end{tikzpicture}


\subsection{}
Hierbei wurden nicht erreichbare Zustände weggelassen.\\
\begin{tikzpicture}[auto, baseline, node distance=27mm, statex/.style={state,draw,minimum width=10mm}]
\node (ts) {\(B \otimes M_{\lnot \phi}\):};
\node[state,accepting,initial,initial text={},below=0mm of ts] (b0m01) {\(b_0, m_0, 1\)};
\node[statex, below=10mm of b0m01]                             (b0m0)  {\(b_0, m_0, 2\)};
\node[statex,right of=b0m0]                                    (b1m0)  {\(b_1, m_0, 2\)};
\node[statex,right of=b1m0]                                    (b2m0)  {\(b_2, m_0, 2\)};
\node[statex,right of=b2m0]                                    (b3m0)  {\(b_3, m_0, 2\)};
\node[statex,right of=b3m0]                                    (b4m0)  {\(b_4, m_0, 2\)};
\node[statex,right of=b4m0]                                    (b5m0)  {\(b_5, m_0, 2\)};
\path[->]
	(b0m01) edge[bend left] node[above right] {\(\{r_1,gr_2\}\)} (b1m0)
	(b0m0)  edge[bend left] node[above]       {\(\{r_1,gr_2\}\)} (b1m0)
	(b1m0)  edge[bend left] node[above]       {\(\{r_1,g_2\}\)}  (b2m0)
	(b2m0)  edge[bend left] node[above]       {\(\{rg_1,r_2\}\)} (b3m0)
	(b3m0)  edge[bend left] node[above]       {\(\{gr_1,r_2\}\)} (b4m0)
	(b4m0)  edge[bend left] node[above]       {\(\{g_1,r_2\}\)}  (b5m0)
	(b5m0)  edge[bend left] node[above]       {\(\{r_1,rg_2\}\)} (b0m0)
;%-path-%
%
\end{tikzpicture}\\
\(L^\omega(A_1 \otimes A_2) \cap L^\omega (M_{\lnot\phi}) = \emptyset\), da der Produkt-Büchi-Automat kein Wort akzeptiert, denn der Endzustand kann nicht unendlich oft durchlaufen werden.
\subsection{}

Da \(L^\omega(A_1 \otimes A_2) \cap L^\omega (M_{\lnot\phi}) = \emptyset\) gilt, erfüllt das System die Spezifikation \(\phi\). 
\end{document}